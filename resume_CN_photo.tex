% !TEX program = xelatex

\documentclass{resume}
\usepackage{graphicx}
\usepackage{tabu}
\usepackage{multirow}
\usepackage[utf8]{inputenc}  
\usepackage{zh_CN-Adobefonts_external} % 
\usepackage{linespacing_fix} % disable extra space before next section
\usepackage{cite}
\usepackage{subfig}   %并排图片宏包
\usepackage{float}    %图片浮动位置

\begin{document}
\pagenumbering{gobble} % suppress displaying page number

\begin{minipage}{0.8\textwidth}
 \fontsize{11pt}{0}{
  姓名:\textbf{李英子} \hspace{3.05cm} 年龄:24  \\
  电话:15799662885 \hspace{2.1cm}  Email:wangyingzics@163.com \\
  \textbf{主修课程}
 }
  \begin{normalsize}
    \begin{itemize}[parsep=0.5ex]
      \item 数学:离散数学、高等xxxx、高等数学、高等xxxx、高等数学
      \item 计算机:计算机xxxx、计算机xxxx(ML)、计算机xxxx(NLP)、计算机xxxx(KG)、计算机xxxx(DM)、数据结构、计算机xxxx、计算机xxxx、计算机网络
    \end{itemize}
  \end{normalsize}
\end{minipage}
\begin{minipage}{0.2\textwidth}
\centering
\includegraphics[height=1.2in,]{resume/pic/pic.jpg}
\end{minipage}

\section{教育背景}
\begin{normalsize}
  \begin{itemize}[parsep=0.5ex]
    \item {2020.09 -- 2023.07}\hspace{1cm}北京邮电大学计算机学院  \hspace{1cm} 计算机科学与技术  \hspace{1.4cm} 硕士
    \item {2015.09 -- 2020.07}\hspace{1cm}北京邮电大学计算机学院  \hspace{1cm} 计算机科学与技术 \hspace{1.4cm} 学士
  \end{itemize}
\end{normalsize}


\section{专业技能}
\begin{normalsize}
  \begin{itemize}
    \item 熟悉 Node.js 开发,熟练使用 npm,webpack,gulp等工具。熟悉前端性能的优化,熟悉使用 Chrome,Safari等前端调试工具。
    \item 熟悉Node.js以及V8的性能和稳定性优化,能对系统整体性能进行评估,解决内存瓶颈。
    \item 熟练使用Git,熟悉团队协作流程,有开源社区交流、Linux运维和部署经验。
    \item 熟悉模块化、前端编译和构建工具,熟练运用主流的移动端JS库和开发框架,并深入理解其设计原理,例如:Zepto、React等。

  \end{itemize}
\end{normalsize}

% \section{\faUsers\ 工作经历}
% \datedsubsection{\textbf{天水师范学院文物与博物馆学系}\hspace{1cm}教师}{2015.09 -- 2019.08}
% \begin{onehalfspacing}
%     \begin{itemize}
%       \begin{large}
%         \item 承担化学与文物、博物馆藏品管理与保护、计算机辅助技术与文物保护、中国文化遗产保护概论等课程教学。
%         \item 赴新疆昌吉州带队支教半年。
%       \end{large}
%     \end{itemize}
% \end{onehalfspacing}


\section{项目经历}
\subsection{{2021.11 -- 2022.02} \hspace{2cm}\textbf{DRG采集系统和数据分析系统-省部级重点项目}}
\fontsize{11pt}{0}{项目描述:采用前后端相结合的技术分析并展示重点人物与重点事件。}
% \subsection{项目描述:采用前后端相结合的技术分析并展示重点人物与重点事件。}
    \begin{normalsize}
      \begin{itemize}
        \item 完成 DRG 采集系统设计, 完成 DRG 数据分析系统设计和其前端开发与部署工作。
        \item 实验室横向课题, 采集医疗数据, 通过数据分析为国家医保投入做技术支持和追溯。
        \item 相关技术: Spring /Vue.js/DevOps 
        \end{itemize}
    \end{normalsize}

\subsection{{2021.08 -- 2021.10} \hspace{2cm}\textbf{施工团队协作系统 Plate 项目负责人-省部级重点项目}}
\fontsize{11pt}{0}{项目描述:采用组件式的新型开发技术分析追踪热点案件的传播路径,预测添加新案件。}
    \begin{normalsize}
      \begin{itemize}
        \item 土木工程管理平台, 用来帮助施工人员进行现场照片采集, 任务发布和工程管理。
        \item 完成原型设计, 数据库设计, 网站脚手架搭建, 响应式布局实现和网站部署等工作。
        \item 相关技术: Node.js/Git/ MySQL  
        \end{itemize}
    \end{normalsize}

\subsection{{2021.06 -- 2021.08} \hspace{2cm}\textbf{国家互联网应急中心云平台管理系统-所级项目}}
\fontsize{11pt}{0}{项目描述:整理不同类别算法,各自进行封装,提供算法调用接口,训练模型并返回模型测试指标。}
    \begin{normalsize}
      \begin{itemize}
        \item 完成 OpenS的虚拟机、数据库和容器等资源, 还包括计费和预警。
        \item 结合使用 xx,实现了 app 登录和个人资料的自动填充,提高了多少的转化率。
        \item 相关技术: Smarty/SVN/SaltStack  
      \end{itemize}
    \end{normalsize}

\subsection{{2021.06 -- 2021.08} \hspace{2cm}\textbf{国家互联网应急中心云平台管理系统-所级项目}}
\fontsize{11pt}{0}{项目描述:整理不同类别算法,各自进行封装,提供算法调用接口,训练模型并返回模型测试指标。}
    \begin{normalsize}
      \begin{itemize}
        \item 完成 OpenS的虚拟机、数据库和容器等资源, 还包括计费和预警。
        \item 结合使用 xx,实现了 app 登录和个人资料的自动填充,提高了多少的转化率。
        \item 相关技术: Smarty/SVN/SaltStack  
      \end{itemize}
    \end{normalsize}

% \section{\faBook\ 学术成果}
% \datedline{\textit{\nth{1} Prize}, Award on xxx }{Jun. 2013}
% \datedline{Other awards}{2015}
% \begin{itemize}[parsep=0.5ex]
%   \begin{large}
%       \item Hongying Zhang, Dawa Shen, Zhiguo Zhang, Qinglin Ma. Characterization of degradation and iron deposits of the wood of Nanhai I shipwreck[J]. Heritage Science, 2022, 10(1): 1-13.  (中科院二区,SCIE和AHCI双检索,IF= 2.843)
%   \end{large}
% \end{itemize}




% \section{\faKey\ 专业培训}
% \begin{itemize}[parsep=0.5ex]
%   \begin{large}
%     \item 2023.04,参加国家文物局中国文化遗产公开课—博物馆研学师资线上培训班,获得证书(NO.202303020237)。
%   \end{large}
% \end{itemize}


\section{奖项荣誉}
% \datedline{\textit{\nth{1} Prize}, Award on xxx }{Jun. 2013}
% \datedline{Other awards}{2015}
\begin{normalsize}
  \begin{itemize}[parsep=0.5ex]
    \item 2023-2024年,北京邮电大学开源软件协会负责人, 组织企业沙龙活动和开源项目  
    \item 2020-2021年,北京邮电大学党委宣传部官微技术部负责人, 负责设计和排版相关工作  
    \item 2020-2021年,获得北京邮电大学“优秀共青团干部”、“北京邮电大学三好学生”。
  \end{itemize}
\end{normalsize}

\section{自我评价}
\begin{itemize}
  \begin{normalsize}
      \item 简历的制作过程考验了一个人的两个能力,逻辑能力和细节能力。写好一份简历,有很多技巧,排版,量化数据等,但有一点最重要的是,自身要有实力。
  \end{normalsize}
\end{itemize}


\end{document}
