% !TEX program = xelatex
\documentclass{resume}
\usepackage{graphicx}
\usepackage{tabu}
\usepackage{zh_CN-Adobefonts_external} % 
\usepackage{linespacing_fix} % disable extra space before next section
\usepackage{cite}

\begin{document}
\pagenumbering{gobble} % suppress displaying page number

\begin{minipage}{0.5\textwidth}
 \fontsize{13pt}{0}{
  姓名:\textbf{英子} \\
  电话:15719752995  \\
  籍贯:四川省成都市
 }
\end{minipage}
\begin{minipage}{0.5\textwidth}
\fontsize{13pt}{0}{
  年龄:30  \\
  Email:yingziyingzi@163.com \\
  政治面貌:中共党员
 }
\end{minipage}

\section{\faGraduationCap\  教育背景}
\subsection{{2019.09 -- 2024.01}\hspace{0.6cm}北京科技大学 \hspace{0.5cm} 科技史与文化遗产研究院 \hspace{0.5cm} 科学技术史专业 \hspace{0.5cm} 博士}

\subsection{{2011.08 -- 2014.01}\hspace{0.6cm}北京科技大学 \hspace{0.5cm} 科技史与文化遗产研究院 \hspace{0.5cm} 科学技术史专业 \hspace{0.5cm} 硕士}



\section{\faUsers\ 工作经历}
\datedsubsection{\textbf{天水师范学院文物与博物馆学系}\hspace{1cm}教师}{2015.09 -- 2019.08}
\begin{onehalfspacing}
    \begin{itemize}
      \begin{large}
        \item 承担化学与文物、博物馆藏品管理与保护、计算机辅助技术与文物保护、中国文化遗产保护概论等课程教学。
        \item 赴新疆昌吉州带队支教半年。
      \end{large}
    \end{itemize}
\end{onehalfspacing}

\datedsubsection{\textbf{天水市博物馆文物保护修复中心}\hspace{1cm}助教} {2014.02 -- 2015.08}
\begin{onehalfspacing}
    \begin{itemize}
        \begin{large}
          \item 从事古旧字画和金属类文物保护修复前的科学分析检测及保护修复工作。
          \end{large}
        \end{itemize}
\end{onehalfspacing}
% }

\section{\faTasks\ 项目经历}
\datedsubsection{\textbf{国家重点项目-海洋出水木质文物铁质沉积物的脱除与控制技术研究}}{2020.12 -- 2024.01}
\begin{onehalfspacing}
    \begin{large}
        \begin{itemize}
          \item 负责工作
        \end{itemize}
    \end{large}
\end{onehalfspacing}
% }
\datedsubsection{\textbf{中国文化遗产研究院中意项目-海洋出水木质文物分析检测、脱盐与填充加固试验工作规程编制}}{2019.01 -- 2020.12}
\begin{onehalfspacing}
    \begin{itemize}
    \begin{large}
      \item 负责工作
    \end{large}
    \end{itemize}
    
\end{onehalfspacing}
% }
\datedsubsection{\textbf{国家文物局重点项目-河南郑韩故城出土金属器及冶铸遗物科学分析研究}}{2012.01 -- 2013.12}
\begin{onehalfspacing}
    \begin{itemize}
      \begin{large}
        \item 负责工作
      \end{large}
    \end{itemize}
\end{onehalfspacing}


\section{\faBook\ 学术成果}
% \datedline{\textit{\nth{1} Prize}, Award on xxx }{Jun. 2013}
% \datedline{Other awards}{2015}
\begin{itemize}[parsep=0.5ex]
  \begin{large}
      \item yingzi Lin, yingzi Ma, yingzi Zhang, yingzi Ma. Characterization of degradation and iron deposits of the wood of Nanhai I shipwreck[J]. Heritage Science, 2028, 10(1)  
      \item yingzi Lin, yingzi Lin, yingzi Zhang, yingzi Ma. Characterization of degradation and iron deposits of the wood of Nanhai I shipwreck[J]. Heritage Science, 2021, 10(1): 1-13.
      \item yingzi Ma, yingzi Lin, yingzi Zhang, yingzi Ma. Characterization of degradation and iron deposits of the wood of Nanhai I shipwreck[J]. Heritage Science, 2022, 10(1): 1-13.
  \end{large}
\end{itemize}

\section{\faCogs\ 专业技能}
\begin{itemize}[parsep=0.5ex]
  \begin{large}
      \item 熟练使用SEM-EDS、FTIR、XRD、XRF、热分析仪、拉曼光谱仪、红外热成像仪、分光测色计、三维视频显微镜、金相显微镜、光学显微镜设备,进行文物科学分析。
      \item 熟练使用Geomagic Wrap、Agisoft Metashape Professional、3Dsmax软件,构建文物高精度模型和彩色模型,进行文物虚拟修复。
      \item 熟练使用PS、CAD软件,通过英语六级,取得考古学及博物馆学高校教师资格证。
  \end{large}
\end{itemize}


\section{\faKey\ 专业培训}
\begin{itemize}[parsep=0.5ex]
  \begin{large}
    \item 2023.04,参加国家文物局中国文化遗产公开课—博物馆研学师资线上培训班,获得证书(NO.202303020237)。
    \item 2020.05,参加中国文化遗产研究院2020年度壁画文物保护修复技术线上培训,获得证书(NO.CACH202000095)。
    \item 2019.03,参加西安交通大学青年教师能力培训。
    \item 2018.12,参加陕西师范大学教师教学能力培训。
    \item 2014.10,参加中国文化遗产研究院文物修复中的化学风险预防—化学清洗方法的选择线下培训,获得证书(NO.20140508020)。
  \end{large}
\end{itemize}


\section{\faStar\ 奖项荣誉}
% \datedline{\textit{\nth{1} Prize}, Award on xxx }{Jun. 2013}
% \datedline{Other awards}{2015}
\begin{itemize}[parsep=0.5ex]
  \begin{large}
    \item 获得国家励志奖学金、研究生一等奖学金、国家助学金,获得“校级三好学生”、“优秀实习生”、“优秀大学生”荣誉称号,博士毕业论文被评为“北京科技大学校级优秀论文”。
    \item 获得“优秀共产党员”、“优秀实习指导教师”荣誉称号。
  \end{large}
\end{itemize}

\section{\faPlus\ 自我评价}
\begin{itemize}[parsep=0.5ex]
  \begin{large}
    \item 本人为人热情真诚,乐观向上,吃苦耐劳;对待工作责任心强,有良好的沟通能力和团队精神;具备一定的科研能力和较为全面的知识体系,能够解决实际研究出现的困难与问题。曾在高校工作多年,具备独立承担教学任务和科学研究的专业知识与技能。
  \end{large}
\end{itemize}


\end{document}
